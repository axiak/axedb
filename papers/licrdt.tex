\title{Locally Invertable Convergent and Commutative Replicated Data Types}
\author{
        Michael Axiak \\
        HubSpot, Inc \\
        Cambridge, MA
}
\date{\today}

\documentclass[12pt]{article}
\usepackage{textcomp}
\usepackage{cmbright}
\usepackage{amsthm}

\newtheorem{mydef}{Definition}

\begin{document}
\maketitle

\begin{abstract}
This is the paper's abstract \ldots
\end{abstract}

\section{Introduction}

Maintaining consistency across a distributed system requires maintaining a global order of
operations or optimistic replication (also known as eventual consistency). Convergent or commutative
replicated data types (CRDT) have been shown to provide a theoretical basis for eventually consistent
systems, rather than ad hoc solutions would could prove to be error prone. \cite{crdt2011}

As CRDTs require no synchronization, they are an attractive building block for highly available
distributed systems. However, they provide a challenge when trying to build an inverted index
to make the data in the distributed system queryable --- when should the data be considered
available for indexing? Typical strategies in building a column index for databases are described
by Abadi {\em et al.} involve a column store index build step that requires freezing the value
for a given key ahead of time. \cite{abadi}

In this paper, the author discusses a strategy for using CRDTs in a two phase database system.
Initially, CRDTs are used to accept writes and serve as the basis or a write optimized store, while
a column store is used for read optimized querying. A strategy for 


\begin{mydef}
Here is a new definition

\begin{itemize}

\item test
\item test2

\end{itemize}
\end{mydef}


Convergent Replicated Data Types (CrRDT) and Commuative Replicated Data Types (CmRDT)
(collectively known as CRDTs) are a useful collection of data structures for highly
available distributed systems aiming for eventual consistency. These data structures
define a {\em semilattice} with a  \cite{crdt2011}

\paragraph{Outline}
The remainder of this article is organized as follows.
Section~\ref{previous work} gives account of previous work.
Our new and exciting results are described in Section~\ref{results}.
Finally, Section~\ref{conclusions} gives the conclusions.

\section{Previous work}\label{previous work}
A much longer \LaTeXe{} example was written by Gil~\cite{crdt2011}.

\section{Results}\label{results}
In this section we describe the results.

\section{Conclusions}\label{conclusions}
We worked hard, and achieved very little.

\bibliographystyle{abbrv}

\begin{thebibliography}{9}

\bibitem{crdt2011}
  Marc Shapiro, Nuno Pregui\c{c}a, Carlos Baquero, Marek Zawirski.
  A comprehensive study of Convergent and Commutative Replicated Data Types.
  [Research Report] RR-7506, 2011, pp.50.
  \textlangle{}inria-00555588\textrangle{}


\bibitem{abadi}
  Daniel Abadi, Peter Boncz, Stavros Harizopoulos, Stratos Idreos, Samnuel Madden.
  The Design and Implementation of Modern Column-Oriented Database Systems.
  Foundations and Trends in Databases, Vol. 5, No. 3 (2012) 197-280.

\end{thebibliography}


\end{document}
This is never printed
